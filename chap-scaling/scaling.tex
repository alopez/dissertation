\chapter{Tera-Scale Translation Models}\label{chap:scaling}

\begin{quote}
	{\em All models are wrong.  Some models are useful.}
	\begin{flushright}
		--George Box
	\end{flushright}
\end{quote}

We've taken translation by pattern matching from a novel
idea to a fully realized implementation and shown that it
is a viable replacement for direct representation of models.
Furthermore, we've solved a difficult algorithmic challenge
in applying it to hierarchical phrase-based translation.
These results match the current state of the art.
Now we seek to improve on this result.  To our
knowledge, none of the past work in this area 
\citep{Callison-Burch:2005:acl,Zhang:2005:eamt} has demonstrated
improvement over a credible baseline.

The lack of prior positive results has encouraged skepticism.
\citet{Zens:2007:hlt-naacl} specifically criticize
the work of \citet{Callison-Burch:2005:acl}.
Their argument centers on two key points.  First, they
claim that the primary advantage posited by that work---
use of longer phrases---does not lead to any improvements in 
translation quality.  They cite both their
own results and those of \citet{Koehn:2003:naacl}.
Second, they object to the memory requirements of
the training data and indices.  Phrase tables
can easily exceed these requirements, but they solve
this by storing their phrase table in an external
prefix tree.  This drastically reduces memory
use.

The external prefix tree is certainly useful.  
However, the argument employed
to promote it overlooks some key points.
First, though it may be true that longer phrases are not 
particularly useful, there are many other axes along which
models can be scaled up.  Second, the memory use
of translation by pattern matching along these axes is constant.
It supports very large models from a single underlying representation.  In 
contrast, an external prefix tree representation must
be generated for each model variant, and 
is proportional to model size.  As model
size increases, this becomes a significant drawback.
We will show that translation by pattern matching enables us
to use models that are impractical to implement using prefix trees.
Since it computes parameters on an as-needed basis, there
is no need to compute or store the entire model.  This
enables us to translate with arbitrarily large models.

There is an additional benefit. Since
the underlying representation is static, it is possible
to experiment with new models simply by modifying algorithmic
parameters or model features.  We never need to recompute a 
full model offline or design specialized extraction utilities.
This makes translation by pattern matching an ideal platform
for model prototyping.

In this chapter we apply translation by pattern matching to the
problem of translation model scaling.  Our experiments focus on
hierarchical phrase-based translation,
since it is superior to standard phrase-based translation 
(cf. \textsection\ref{sec:overview-results}, \textsection\ref{sec:hiero-results}).
We first illustrate the rapid prototyping capabilities of translation
by pattern matching (\textsection\ref{sec:prototyping}).
We then explore several scaling axes---some obvious and 
some novel---and show that we can substantially improve on 
the baseline system (\textsection\ref{sec:scaling-works}).  Our
experiments highlight interesting characteristics of the model
and interactions of various components.
Finally, we conclusively demonstrate that reproducing our results
with phrase tables would be highly impractical
(\textsection\ref{sec:tera-scale-model}).

\section{Rapid Prototyping via Pattern Matching}\label{sec:prototyping}

Translation by pattern matching shortens
experimental cycles by removing time-consuming offline model computation.
Consider the conventional architecture illustrated in Figure~\ref{fig:table-based}.
It requires us to re-extract rules and recompute
all model parameters for each new model variant. This process can
require several hours or even days.  Translation 
by pattern matching enables us to explore many model variants
because all models use the same underlying representation. 
It is generated once, often in a matter of minutes
(Figure~\ref{fig:prototyping}).

\figpreamble
\begin{figure}
	\figfontsize{
	\begin{center}
		\begin{center}
	\begin{tikzpicture}
%		\node (mem)[rectangle,rounded corners,fill=lightgray,minimum height=5.0cm,minimum width=2.8cm,label=90:main memory] at (10,-0.75){};
		\node (corpus)[rectangle,draw,label=90:parallel text,minimum height=1.5cm,minimum width=1.5cm] at (0,0) {};
		\foreach \y in {0.05,0.1,...,1.4}{
			\draw[yshift=-0.7cm] (-0.65,\y) -- (-0.05,\y);
			\draw[yshift=-0.7cm] (0.05,\y) -- (0.65,\y);
		}

%		\node (rules)[rectangle,draw,label=90:extracted rules,minimum height=2.5cm,minimum width=0.75cm] at (5,0) {};
%		\foreach \y in {0.05,0.1,...,2.4}{
%			\draw[xshift=5.0cm,yshift=-1.2cm] (-0.3,\y) -- (-0.05,\y);
%			\draw[xshift=5.0cm,yshift=-1.2cm] (0.05,\y) -- (0.3,\y);
%		}

		\foreach \tabley/\id in {-2.5/3, 0/2, 2.5/1}{
		\begin{scope}[xshift=7cm,yshift=\tabley cm]
			\node (pt)[rectangle,draw,fill=white,minimum height=2.0cm,minimum width=2.20cm] at (0,0) {};
			\foreach \y in {0.05,0.1,...,1.9}{
				\draw[yshift=-0.95cm] (-1.0,\y) -- (-0.75,\y);
				\draw[yshift=-0.95cm] (-0.65,\y) -- (-0.4,\y);
				\draw[yshift=-0.95cm] (-0.3,\y) -- (-0.05,\y);
				\draw[yshift=-0.95cm] (0.05,\y) -- (0.3,\y);
				\draw[yshift=-0.95cm] (0.4,\y) -- (0.65,\y);
				\draw[yshift=-0.95cm] (0.75,\y) -- (1.0,\y);
			}
			\node (process) [rectangle,rounded corners,draw,text width=2.5cm,text centered] at (-3.5,0) {rule extraction and scoring (variant \id)};
			\node (decoder) [rectangle,rounded corners,draw] at (4,0) {decoder};

			\draw[->] (corpus) -- (process);
			\draw[->] (process) -- (pt);
			\draw[->] (pt) -- (decoder);
		\end{scope}
		}
		\node [anchor=south,above=5mm] at (pt.north) {phrase tables};

		\draw[snake=brace,raise snake=5mm,segment amplitude=2mm] (process.north west) -- (process.north east) node [pos=0.5,anchor=south,above=7mm] {hours or days};

		\draw[snake=brace,raise snake=5mm,segment amplitude=2mm] (decoder.north west |- process.north) -- (decoder.north east |- process.north) node [pos=0.5,anchor=south,above=7mm] {hours};

		\node [anchor=south west] at (-1cm,-4.3cm) {Conventional translation with phrase tables};
		\node [anchor=north west] at (-1cm,-8.5cm) {Translation by pattern matching};
		\draw (-1cm,-4.3cm) -- +(13cm,0cm);

		\begin{scope}[yshift=-8cm]
			\node (corpus)[rectangle,draw,label=90:parallel text,minimum height=1.5cm,minimum width=1.5cm] at (0,0) {};
			\foreach \y in {0.05,0.1,...,1.4}{
				\draw[yshift=-0.7cm] (-0.65,\y) -- (-0.05,\y);
				\draw[yshift=-0.7cm] (0.05,\y) -- (0.65,\y);
			}

			\begin{scope}[xshift=7cm]
				\node (compiled corpus)[rectangle,draw,minimum height=1.5cm,minimum width=1.5cm] at (0,0) {};
				\foreach \y in {0.05,0.1,...,1.4}{
					\draw[yshift=-0.7cm] (-0.65,\y) -- (-0.05,\y);
					\draw[yshift=-0.7cm] (0.05,\y) -- (0.65,\y);
				}
			\end{scope}
			\node [anchor=south,text width=3.5cm,text centered] at (compiled corpus.north) {indexed, memory-mapped parallel text};
			\node (process) [rectangle,rounded corners,draw,text width=2.5cm,text centered] at (3.5,0) {compilation and indexing};
			\draw[->] (corpus) -- (process);
			\draw[->] (process) -- (compiled corpus);
			\foreach \decodery/\id in {-1.5/3,0/2,1.5/1}{
				\node (decoder) [rectangle,draw,rounded corners,text centered,text width=1.75cm] at (11,\decodery) 
					{decoder (variant \id)};
				\draw[->] (compiled corpus) -- (decoder.west);
			}
			\draw[snake=brace,raise snake=5mm,segment amplitude=2mm] (process.north west) -- (process.north east) node [pos=0.5,anchor=south,above=7mm] {minutes or hours};
			\draw[snake=brace,raise snake=5mm,segment amplitude=2mm] (decoder.north west) -- (decoder.north east) node [pos=0.5,anchor=south,above=7mm] {hours};
		\end{scope}
	\end{tikzpicture}
\end{center}

	\end{center}}
	\figpostamble
	\label{fig:prototyping}
	\caption{Experimental workflow for conventional architecture vs. translation by pattern matching.}
\end{figure}

Generating a conventional system using an external prefix tree 
\citep{Zens:2007:hlt-naacl} takes a few steps.

\begin{enumerate}
	\item Extract all rule instances from the corpus.
	\item Sort and score the rule instances.
	\item Write the prefix tree a memory-mapped file.
\end{enumerate}

\noindent In many experimental scenarios, we must redo these steps for
each system variant.  For instance, to experiment with different maximum
phrase lengths, we must generate a new phrase table for each length.
Furthermore, complexity depends on the size of 
the model.  The time needed to sort and score rules increases with
their number, so a model that generates more rules will take longer
to create.  In contrast, generating a pattern matching system requires the following
steps.

\begin{enumerate}
	\item Write the source training data as a memory-mapped suffix array.\footnote{For 
		suffix array construction we use an algorithm due to 
		\citet{Larsson:1999:tr}. \citet{Puglisi:2007:csur} 
		identify several faster algorithms.}
	\item Write the target training data as a memory-mapped array.
	\item Write the alignment as a memory-mapped array.
	\item Write the lexical weighting table as a memory-mapped data structure.
	\item Write precomputed collocations and inverted indexes as memory-mapped data structures.
\end{enumerate}

\noindent The complexity of this offline initialization is mostly
model-independent.  Even when it is not, its modularity is a benefit.
For instance, to use a different word alignment of the same
corpus, we need only generate a new alignment data structure.  To move
from a standard phrase-based model to a hierarchical model, we only
need to generate the precomputed collocations and inverted indices.
The cost of generating these modular components is minor.

To make matters concrete, we measured the generation time for both representations of our
baseline hierarchical model (\textsection\ref{sec:hiero-results}).
We generated a conventional system using an external prefix tree and a pattern
matching system (precomputing for the 100 most frequent patterns).  The
results are given in Table~\ref{table:startup-time}.  The size of
each representation is illustrated in Table~\ref{table:representation-size}
The prefix tree representation takes nearly 90 times as long to produce
and requires over 7 times as much memory as our indirect representation.

\figpreamble
\begin{table}
	\begin{center}
		\begin{tabular}{llr}
	\multicolumn{2}{l}{Method} & CPU time (seconds) \\ \hline
	\multicolumn{2}{l}{\bf Phrase tables (external prefix tree)} & {\bf 39051}\\
	~ & Phrase extraction & 31771\\
	~ & Phrase table scoring & 5322\\
	~ & External prefix tree generation & 1958\\ \\
	\multicolumn{2}{l}{\bf Translation by pattern matching} & {\bf 445} \\
	~ & Suffix array construction (source) & 100\\
	~ & Array construction (target) & 40\\
	~ & Precomputations & 126\\
	~ & Array construction (alignments) & 96\\
	~ & Lexical weight computation & 73\\
\end{tabular}

	\end{center}
	\figpostamble
	\caption{CPU time needed to generate the baseline model.}
	\label{table:startup-time}
\end{table}

If we had a sufficiently large cluster, we might reduce the computation
of phrase tables to a few hours using distributed 
methods.\footnote{Chris Dyer, personal communication.}  However,
this is still dependent on model complexity.
We now turn to the topic of scaling our models to a size
that dramatically outstrips our ability to 
represent them using phrase tables.

\figpreamble
\begin{table}
	\begin{center}
		\begin{tabular}{llr@{~}l}
	\multicolumn{2}{l}{Phrase Tables (external prefix tree)} & & \\
	~ & Rule instances extracted & 195 & million \\
	~ & Unique rules & 67 & million\\
	~ & Compressed extract file size & 1.4 & GB \\
	~ & Uncompressed extract file size & 9.3 & GB \\
	~ & {\bf Representation size} & {\bf 6.1} & {\bf GB} \\
	\multicolumn{2}{l}{Translation by Pattern Matching} & & \\
	~ & Source-side suffix array size & 343 & MB \\
	~ & Target-side array size & 123 & MB\\
	~ & Alignment size & 119 & MB \\
	~ & Lexical weight table size & 27 & MB \\
	~ & Precomputation size (top 100) & 240 & MB \\
	~ & {\bf Total representation size} & {\bf 852} & {\bf MB} \\
\end{tabular}


	\end{center}
	\figpostamble
	\caption{Size of representations for the baseline model.}
	\label{table:representation-size}
\end{table}

\section{Experiments}

We exploited the rapid prototyping capabilities of our
system to run a large number of experiments.  Our experiments
focus on scaling, but they also highlight
some interesting characteristics of the 
underlying model.

\subsection{Relaxing Grammar Length Restrictions}\label{sec:longer-phrases}

\citet{Callison-Burch:2005:acl} and \citet{Zhang:2005:eamt} focused on
two specific scaling axes for translation models: larger corpora and
longer phrases.  We generalize these ideas under the rubric of adding
more data, and generating more rules from available data.  We examine
the former in \textsection\ref{sec:scaling-works} below.  Here
we are concerned with the latter.  In hierarchical
phrase-based translation, the analogue to an increase in maximum
phrase length is relaxation of the grammar length restrictions
(\textsection\ref{sec:hierarchical-translation}).
We experimented with several such restrictions.

\figpreamble
\begin{table}
	\begin{center}
			(a) \begin{tabular}{cc}
		$\maxphrasespan$ & BLEU \\ \hline
		1 & 16.1 \\
		2 & 23.8 \\
		3 & 25.5 \\
		4 & 28.1 \\
		5 & 28.5 \\
		6 & 30.3 \\
		7 & 30.3 \\
		8 & 30.5 \\
		9 & 30.8 \\
		\bf{10} & \bf{30.9} \\
		11 & 31.1 \\
		12 & 31.0 \\
		13 & 31.2 \\
		14 & 31.2 \\
		15 & 30.8 \\
	\end{tabular}
	~~~(b) \begin{tabular}{cc} 
		$\maxphraselen$ & BLEU \\ \hline
		1 & 16.9 \\
		2 & 25.4 \\
		3 & 29.7 \\
		4 & 30.4 \\
		\bf{5} & \bf{30.9} \\
		6 & 30.9 \\
		7 & 30.8 \\
		8 & 30.8 \\
		9 & 30.9 \\
		10 & 30.7 \\
		~\\
		~\\
		~\\
		~\\
		~\\
	\end{tabular}


	\end{center}
	\figpostamble
	\caption{Effect of varying phrase span and phrase length parameters.  
		Baseline values are in bold.}
		\label{table:length-limits}
\end{table}

First, we considered the effect of the maximum phrase span ($\maxphrasespan$)
for rules extracted from training data.  Recall that this parameter
limits the span of phrases that can be extracted 
from the training data.  To see how this
affects our ability to translate, suppose that an input
sentence contains a source phrase $uXv$.  It is possible
that subpatterns $u$ and $v$ are collocated in a training
sentence, but within a window longer than the maximum
phrase span.  In this case, the restriction prevents us
from learning a translation rule that translates the 
collocated patterns as a single unit.  Relaxing the span
limit increases the chance of finding translation rules for
collocated source phrases, or for finding additional examples
of such rules.  The risk is an increased chance of learning
translation rules for otherwise meaningless collocations.  
\citet{Chiang:2007:cl} fixes the limit at 10.  To paint a complete picture
of the effect of this parameter, we considered all values
between 1 and 15.  Results are reported in 
Table~\ref{table:length-limits} (a).  We found that accuracy
plateaus just below the baseline setting.

Next, we considered the effect of the maximum phrase 
length ($\maxphraselen$), which limits the number of symbols 
that can appear in a source phrase.
This is the closest analogue of the maximum
phrase length in a standard phrase-based model.  By
increasing this parameter, we increase the number of words
(either contiguous or in collocated subpatterns) that can be
translated as a single unit.  As with the span parameter, we
also run the risk of translating words based on purely
coincidental collocation.  We considered values between 1 and 10.
Results are reported in 
Table~\ref{table:length-limits} (b).  As with the 
previous experiment, we found that accuracy
plateaus just at the baseline setting.

%Finally, we considered the effect of decreasing the minimum
%gap length.  Increasing this parameter reduces
%the number of possible subphrases that can be subtracted from
%larger phrases to form gaps.  In turn, this reduces the number
%of rules that can be learned from a single source phrase in the
%training data, and reduces grammar size.
%\citet{Chiang:2007:cl} uses a value of two.  Since grammar size
%is not a concern in our implementation, we reduce it to one,
%allowing single words to be subtracted as subphrases.  This enables
%the system to learn translation rules for patterns
%$uXv$, in which subpatterns $u$ and $v$ are separated by a single
%word.  In a single experiment, we found that reducing the minimum
%gap size resulted in a BLEU score of 30.5, which does not improve
%the baseline system score of 30.9.

The results here essentially extend the findings of \citet{Koehn:2003:naacl}
and \citet{Zens:2007:hlt-naacl} to the hierarchical case.  Though
scaling along these axes is unhelpful, there is still a large space
for exploration and improvement.

\subsection{Interlude: Hierarchical Phrase-Based Translation and Lexical Reordering}
\label{sec:lexicalized-reordering}

Along a related line of inquiry, we considered the effect of increasing the number of 
nonterminals in translated rules.  Although \citet{Chiang:2007:cl}
limits the number of nonterminals to two, his decoder implementation
can translate rules with arbitrary numbers of nonterminal symbols.
For completeness, we also consider the effect of reducing
this parameter to one or zero.  We also extend this experiment
to gain insight into the benefits of hierarchical model over a
conventional phrase-based model.

A popular hypothesis posits that 
hierarchical phrase-based translation is essentially
equivalent to {\em lexicalized reordering} in conventional phrase-based models
\citep{Tillman:2004:hlt-naacl,Koehn:2005:iwslt,Al-Onaizan:2006:acl-coling}.\footnote{
This hypothesis was suggested to me
independently in personal communications with several researchers, 
including Chris Callison-Burch, Chris Dyer, Alex Fraser, and Franz Och.}
Lexicalized reordering models parameterize reordering decisions on the
words or phrases being translated.  Proponents of this hypothesis
argue that, since the synchronous grammar encodes reordering decisions
along with lexical translations, it performs the same function.  However,
this differs from the typical justification for models that translate
such phrases \citep[e.g.]{Chiang:2007:cl,Simard:2005:hlt-emnlp,Quirk:2006:hlt-naacl}.
They argue that the ability to translate discontiguous phrases is central
to the power of these models.  If this is not true, then the focus on
discontiguous phrases may be a dead end, since, as we have seen, their
rulesets are quite unwieldy.  We can instead simply pursue phrase-based
translation using lexicalized distortion, since its representation is
more compact.

To tease apart these claims, we make the 
following distinction.  Hierarchical phrases
containing a single subpattern---that is,
phrases in the form $u$, $Xu$, $uX$, or $XuX$---are
interchangeable with lexicalized reordering in a conventional
phrase-based model.  In contrast, hierarchical phrases
representing discontiguous units---minimally $uXv$---encode
translation knowledge that is strictly outside the purview
of lexical reordering, specifically the translation of collocated
patterns as a single unit.  This delineation is arguably too fine,
but for scientific purposes it is good first approximation,
allowing us to cleanly interpret the results.

To test the lexical reordering hypothesis, we implemented a
restriction on the number of subpatterns that could appear in either
the source or target side of a hierarchical phrase.  With a limit
of one, only one subpattern can appear in a hierarchical phrase, though
the phrase may contain between zero and two nonterminals.\footnote{
Note that this differs from a restriction on the number of 
nonterminals, which by definition must appear in equal number in both
the source and target.  This is not so with subpatterns.  For instance,
the rule $X \longleftarrow \cidx{X}{1}u / v\cidx{X}{1}w$ is a legal 
SCFG rule, though it contains only one source subpattern 
and two target subpatterns.  Therefore we apply
the restriction independently to both source and target phrases.}
We ran several
experiments varying both the number of subpatterns and the
number of nonterminals.  Results are in 
Table~\ref{table:lexicalized-reordering}.
For comparison, we also include results of the
phrase-based system Moses \citep{Koehn:2007:acl-demo} with and
without lexicalized reordering.\footnote{This is a different
decoder than the one we modified in \textsection\ref{sec:overview-results},
which does not support lexicalized reordering.  The results with
the Moses decoder are slightly better overall.}

\figpreamble
\begin{table}
	\figfontsize{
	\begin{center}
		\begin{tabular}{ccccc}
	Subpatterns & Nonterminals & Max Pattern & BLEU & Remark\\ \hline
	1 & 0 & $u$ & 26.3 & monotone \citep[cf.][]{Chiang:2007:cl}\\
	-- & -- & -- & 29.4 & Moses, no lexicalized reordering \\
	-- & -- & -- & 30.7 & Moses with lexicalized reordering \\
	1 & 1 & $uX,Xu$ & 30.2 & lexicalized reordering equivalent\\
	1 & 2 & $XuX$ & 30.0 & lexicalized reordering equivalent\\
	2 & 1 & $uXv$ & 30.5 \\
	2 & 2 & $uXvX,XuXv$ & 30.8 \\
	3 & 2 & $uXvXw$ & 30.9 & Hiero \citep{Chiang:2005:acl,Chiang:2007:cl}\\
	4 & 3 & $uXvXwXy$ & 30.9 \\
	5 & 4 & $uXvXwXyXz$ & 30.8 \\
\end{tabular}

	\end{center}}
	\figpostamble
	\caption{Effect of varying the maximum number of nonterminals and subpatterns.}
		\label{table:lexicalized-reordering}
\end{table}

Our results are consistent with those found
elsewhere in the literature.  The most strict setting
allowing no nonterminal symbols replicates a result in 
\citet[Table~7]{Chiang:2007:cl}, with
significantly worse accuracy than all other models.
The most striking result is that the accuracy of Moses
with lexicalized reordering is indistinguishable from the accuracy
of the hierarchical system.  Both improve over non-lexicalized
Moses by about 1.4 BLEU.  However, the
hierarchical emulation of lexicalized reordering shows that
the full hierarchical model owes its power only partly to
this effect.  Additional improvement is seen when we add
discontiguous phrases.  That the effect of lexicalized reordering is
weaker in the hierarchical model is unsurprising, since its
parameterization is much simpler than the one used by the
Moses, which includes several specialized features for this
purpose.  This suggests that the hierarchical model could
be improved through better parameterization, and still
benefit from the translation of discontiguous phrases.

Finally, we observe that using more than two nonterminals does
not improve the hierarchical model.

\subsection{Complementary Scaling Axes}\label{sec:scaling-works}

So far, we have extended the results of \citet{Zens:2007:hlt-naacl}
and \citet{Koehn:2003:naacl} to hierarchical phrase-based
translation, confirming that longer phrases do not substantially
improve these systems.  We have also shed some light on the
interpretation of hierarchical phrase-based translation.
While these are interesting scientific
findings, they don't improve our baseline system.
We now investigate other axes
along which we can scale our models, some obvious and some novel.

Because we find these axes to be complementary, we present
them together in this section, along with results 
showing improved translation
accuracy.  Our improved system incorporates a large amount of 
out-of-domain data (\textsection\ref{sec:out-of-domain}),
sparse discriminatively trained word alignments 
(\textsection\ref{sec:discriminative-alignment}),
loose phrase extraction heuristics 
(\textsection\ref{sec:extraction-heuristics}), and 
a novel parameterization enabled by our approach
(\textsection\ref{sec:coherence}).  These factors
represent a synthesis of recent trends in statistical
machine translation and lead to substantial improvement
(\textsection\ref{sec:scaling-results}).  
In \textsection\ref{sec:tera-scale-model} we will show
that this synthesis could not have been done without 
translation by pattern matching.

\subsubsection{Out of Domain Data}\label{sec:out-of-domain}

Translation by pattern matching enables us to train from very 
large corpora.  In our experiments, we were not able to find
corpora that were too large to comfortably fit in memory on our
research machines, each of which had 8GB RAM.
To our knowledge, the training corpus of Chinese-English system
that we used in our baseline systems is among the largest corpora
of processed {\em in-domain} training data currently in use---
that is, the genre of this training data is identical to that
of the test data.\footnote{This corpus contains nearly all of the parallel 
newswire data available from the Linguistic Data Consortium.}  To move
to larger corpora, we added a large quantity of {\em out-of-domain data}
---data drawn from a different genre than the test, though in 
the same language pairs.  We used a large parallel text of 
United Nations proceedings.  It is almost three times
the size of our in-domain training corpus, and makes our full 
training corpus among the largest assembled for Chinese-English 
translation tasks.\footnote{Our corpus is over half the size of
one used by \citet{Zens:2007:hlt-naacl} for a Chinese-English
task, and comparable in size
to those used by \citet{Fraser:2006:acl-coling} for Arabic-English
and French-English tasks.} Corpus statistics
are reported in Table~\ref{table:corpus-statistics}.  The corpus
was tokenized, aligned, and otherwise preprocessed separately from
the newswire corpus using identical tools.

Making the best use of out-of-domain data in statistical 
machine translation is a difficult problem that has only 
recently received any attention \citep{foster:2007:smt}.
The simplest strategy is to simply concatenate the out-of-domain
data with the in-domain data.  This
raises a concern.  A source phrase with a different
sense in the out-of-domain data may occur so frequently
that it outweighs the correct sense found in the 
in-domain data.  Therefore, we treat each corpus independently.  
From each corpus, we sample up to 300 rule
occurrences, and compute both lexical weighting and the
source-to-target phrase translation probability separately
on each sample.  For the UN corpus, the resulting probabilities
are incorporated into three new features.  These features
receive a value of zero for any rule computed from the newswire
data.  Likewise, the original source-to-target phrase
translation probability and lexical weighting features receive a value
of zero for rules computed from the UN data.  This is not 
as elegant as the model used by \citet{foster:2007:smt} 
but is similar and easy to implement in our setting.

\figpreamble
\begin{table}
	\begin{center}
	\begin{tabular}{cccc}
		Corpus & Sentences & Chinese tokens & English tokens \\ \hline
		Newswire & 1.02 million & 27 million & 28 million \\
		UN & 2.53 million & 82 million & 79 million \\
		Combined & 3.55 million & 107 million & 107 million \\
	\end{tabular}
	\end{center}
	\figpostamble
	\caption {Sizes of experimental corpora.}
	\label{table:corpus-statistics}
\end{table}


\subsubsection{Discriminatively Trained Word Alignments}
\label{sec:discriminative-alignment}

The second axis that we vary concerns the underlying word alignment.
Two trends are apparent in recent word alignment research.
First, there has been substantial success in discriminative
training of word alignments as measured by intrinsic
criteria.  Second, there is considerable debate over whether
these improvements carry over to the translation task itself
\citep{Ayan:2006:acl-coling,Fraser:2007:cl,Lopez:2006:amta}.

A discriminative aligner that improves in both intrinsic
measures and translation tasks is the
maximum entropy aligner of \citet{Ayan:2006:hlt-naacl}.
Their alignment improves translation accuracy from 24.0
to 25.6 BLEU over a standard aligner on the NIST 2003
Chinese-English task.  This is a significant improvement.
However, both the baseline and the improved system are
far from state-of-the-art due to the relatively small training
set of 107 thousand sentences.\footnote{For comparison,
our baseline system achieves a score of 32.9 BLEU on this task, 
though we use it as a development corpus.}  The newswire 
corpus used in our experiments is ten times larger, and the
full corpus including out-of-domain data is 35 times 
larger.  Experience has shown that improvements found in
small data conditions often do not hold in large data 
conditions \citep{Fraser:2007:cl,Lopez:2006:amta},
so we believe this is important to test.
\citet{Ayan:2006:hlt-naacl} also used a quite
strict phrase length setting due to computational constraints.
These constraints were disk space and offline
computation time, which are not a concern our system
due to its small footprint in both disk use and CPU
time.

For comparison, we also considered the heuristically
combined GIZA++ alignments used by our baseline system.
In both cases, the in-domain and out-of-domain corpora
were aligned separately.

\subsubsection{Phrase Extraction Heuristics}
\label{sec:extraction-heuristics}

Translation by pattern matching becomes extremely relevant
when we consider the interaction of word alignment and
phrase extraction heuristics.  The GIZA++ alignments
used in our baseline system are dense
alignments.  The number of alignment links is over 96\%
of the number of word tokens in each language.  The actual
percentage of aligned words may be somewhat less than this,
since some words may participate in multiple alignments.
However, we can safely assert that most words are aligned.
Recall that the tight extraction heuristic requires aligned words
at both the edges of main phrases and subtracted phrases
(\textsection\ref{sec:source-driven-rule-extraction},
\textsection\ref{sec:hierarchical-extraction}).
Since most words are aligned, this heuristic filters out
relatively few candidate phrases.  Conversely, the loose
heuristic, which allows phrases to contain unaligned words
at the borders of aligned phrases, will result in only a
small number of additional extracted phrases.  

In contrast, the maximum entropy alignments are sparse.
The number of alignment links is less than 70\% of the
number of word tokens in each language.  These alignments
interact strongly with the extraction heuristics.
Consider a phrase $uXv$ containing two collocated
subpatterns and a single gap.  We will make the
coarse assumption that any given word has only a 70\%
chance of being aligned.  Since the tight heuristic requires
that both the main phrase and the subphrase have aligned 
edge words, the chance that an arbitrary phrase meets
this criterion is less than 25\%.  The situation is even
worse for phrases with two gaps.  In contrast, the loose
heuristic not only removes this restriction, but permits
the extraction of many variant translations containing
unaligned words at the edges of the main phrase and subtracted
phrases.  This results in the extraction of a very 
large inventory of phrases.
\citet{Ayan:2006:acl-coling} show that this
interaction dramatically affects translation performance.  
They found that using the loose heuristic with sparse alignments
improved accuracy by 3.7 BLEU over the tight heuristic.  
This finding is supported by \citet{Fraser:2007:cl} 
and \citet{Lopez:2006:amta}, who found that larger phrase
inventories tend to improve translation accuracy, particularly
for Chinese to English translation.

Scaling this result to a very large dataset 
is a natural application for translation
by pattern matching, since it implies that a very large
number of phrases would have to be processed if we constructed
a phrase table offline.  Indeed, \citet{Ayan:2006:acl-coling}
do not provide results for some more permissive settings due
to computational cost, even in their small data setting.  
Later, we will show that the interaction of loose extraction
heuristics on sparse alignments on a large corpus produces
a very large model \textsection\ref{sec:tera-scale-model}.

\subsubsection{Coherent Translation Units}
\label{sec:coherence}

An interesting modification to our
baseline model is enabled by pattern matching. 
Recall from \textsection\ref{sec:overview-baseline}
that for every phrase, we compute a marginal for 
phrase translation probabilities using the number of
rules containing the source phrase.

We saw in \textsection\ref{sec:source-driven-rule-extraction}
and \textsection\ref{sec:hierarchical-extraction} that
several conditions can prevent extraction of a translation
for any particular occurrence of a source phrase.
The target phrase
might be too long, there might be unaligned words
at its edges if we are using a tight extraction heuristic,
the reflected source span might not match the main
phrase, or one of the subtracted phrases violates
these restrictions.

Translation by pattern matching gives us the ability
to incorporate the knowledge that we gain from
failed phrase extraction into our model.
Informally, we think of this as follows.  If a
phrase occurs frequently in our corpus, but we are
often unable to extract a translation for it,
then our confidence that it represents a 
coherent unit of translation is diminished.  Conversely,
if we are usually able to extract a translation of the phrase, then
we believe that the phrase is a natural unit of translation.
We call this property {\em coherence}.  Note that this
property is not captured in the conventional offline
extraction method in which only valid phrase 
pairs are extracted.  If a phrase occurs many times but
we can only extract a translation for it a few 
times, then those translations receive very high probabilities,
even though they may only reflect coincidence or even 
misalignments.

We define coherence as the ratio of successful extractions 
to attempted extractions.  This is added as a feature in our
log-linear model.  As an alternative, we can incorporate the
notion of coherence directly into the phrase translation
probability.  We call this the coherent phrase-to-phrase
translation probability.  We found in preliminary experiments
that these alternatives performed equally well.  In the 
interest of keeping our model as simple as possible, we choose
the latter option since it does not increase the number of 
features.

\subsubsection{Results}\label{sec:scaling-results}

\figpreamble
\begin{table}
	\figfontsize{
	\begin{tabular}{llcc}
		\multicolumn{2}{l}{System} & BLEU & loss\\ \hline
		\multicolumn{2}{l}{Best (all modifications)} & 32.6 & --\\
		~ & with grow-diag-final-and alignment instead of maximum entropy alignment & 32.1 & -0.5 \\
		~ & with tight extraction heuristic instead of loose & 31.6 & -1.0 \\
		~ & without UN data & 31.6 & -1.0\\
		~ & without separate UN grammar & 32.2 & -0.4 \\
		~ & with standard $p(f|e)$ instead of coherent $p(f|e)$ & 31.7 & -0.9 \\
		\multicolumn{2}{l}{Baseline (Phrase Tables)} & 30.7 & -1.9 \\
		\multicolumn{2}{l}{Baseline (Pattern Matching)} & 30.9 & -1.7 \\
	\end{tabular}}
	\figpostamble
	\caption{Results of scaling modifications and ablation experiments.}
	\label{table:scaling-results}
\end{table}

As in \textsection\ref{sec:overview-results} and 
\textsection\ref{sec:hiero-results}, each system configuration
was optimized separately on the development set.
We repeat the baseline system results from \textsection\ref{sec:hiero-results}
and show the results for a system combining all of the modifications
described above.  In addition, we performed ablation experiments
to study the contribution of each modification to the improved
system (Table~\ref{table:scaling-results}).  
Our result improves significantly over the baseline.  The
improvement over the conventional baseline is 1.9 BLEU.  Since
this is a strong baseline, our improvements are
quite substantial.  We find that
all the modifications described above contribute in some way to the
improvement.  The results highlight the complementary nature of our 
modifications.  In some cases the removal of even one of them
causes a loss of over one BLEU point.

To confirm these results, we ran two additional experiments.
First, we compared the baseline and improved systems on the NIST 2006 
Chinese-English task.  Since we did not perform any other experimentation
on this dataset it serves as a validation set.  The baseline system
scores 27.0, while our modified system improves this to 28.4,
a statistically significant improvement.

We also compared our modified system against an augmented baseline
using a 5-gram language model and rule-based number translation.  The objective
of this experiment is to ensure that our improvements are complementary to
better language modeling, which often subsumes other improvements.
The new baseline achieves a score of 31.9 on the NIST 2005 set, making
it nearly the same as the strong results reported by 
\citet{Chiang:2007:cl}.  Our modified system increases this to 
34.5, a substantial improvement of 2.6 BLEU.

\section{A Tera-Scale Translation Model}\label{sec:tera-scale-model}

To get a sense of the what our improvements would require
if we attempted them in a system based on direct representation,
we estimated size of the synchronous grammar table
that would be equivalent to the configuration of our best
system.  Due to constraints on disk space and time, we did not
construct the table itself, for reasons that will become
clear below.  We modified
the extractor to simply report the number of rules that it would
normally extract under our model setting.  To estimate 
phrase table size, we assume
that the disk space requirements and the number of scored rules
grows linearly in the number of extracted rules.\footnote{This
is perhaps conservative, since our corpus is heterogeneous, and
since the ratio of unique rules tends to increase with their
overall number.}
Therefore, we simply multiply the numbers in 
Table~\ref{table:startup-time} by the ratio of extracted
rules between the improved and baseline systems.

The number of extracted rules and the resulting estimates
of phrase table size (Table~\ref{table:estimated-phrase-table})
show that it would be completely impractical to compute such
a phrase table.  The large corpus, sparse alignments, 
and loose extraction heuristics combine to produce a number
of rule occurrences that is two orders
of magnitude larger than in the baseline system.  Merely counting
all of these occurrences took almost five days on our 17-node
research cluster.  We estimate that the resulting files would
require about a terabyte of disk space.  Assuming that we even
had adequate external algorithms to sort, score, 
and generate a memory-mapped prefix tree for such 
large files, we hesitate to estimate how
long it would take.  This estimate does not even address
the question of computing the coherent phrase translation
probabilities in an offline setting.  In order to compute this feature using offline methods, we would need to extract an event for each occurrence of a pattern that might be considered a legal phrase, even if the occurrence was not part of an extractable phrase pair.  This would substantially inflate the number of events extracted from the corpus and increase the amount of offline computation beyond the estimate reported here.

By comparison, it took 1.3 hours to generate the data files for
our pattern matching decoder on a single CPU, and we completed
the full complement of experiments in Table~\ref{table:scaling-results}
in substantially less time than it took to simply count the 
rules that would be generated in the conventional system.

\figpreamble
\begin{table}
	\begin{center}
		\begin{tabular}{llr@{~}l}
	\multicolumn{2}{l}{Phrase Tables (external prefix tree)} & & \\
	~ & Extract time & \bf{77} & \bf{CPU days} \\
	~ & Actual rule instances extracted & 19.3 & billion \\
	~ & Estimated unique rules & 6.6 & billion\\
	~ & Estimated compressed extract file size & 135 & GB \\
	~ & Estimated uncompressed extract file size & 917 & GB \\
	~ & Estimated total representation size & \bf{604} & \bf{GB} \\
	\multicolumn{2}{l}{Translation by Pattern Matching} & & \\
	~ & Total generation time & \bf{1.3} & \bf{CPU hours} \\
	~ & Source-side suffix array size & 1.36 & GB \\
	~ & Target-side array size & 461 & MB\\
	~ & Alignments size & 311 & MB \\
	~ & Lexical weight table size & 63 & MB \\
	~ & Precomputation size (top 100) & 1.24 & GB \\
	~ & Total representation size & \bf{3.4} & \bf{GB} \\
\end{tabular}





	\end{center}
	\figpostamble
	\caption{Generation times and size of representations for 
		the improved model (cf. Table~\ref{table:representation-size}).}
	\label{table:estimated-phrase-table}
\end{table}

\section{Conclusions}

Our results demonstrate that translation by pattern
matching enables practical, rapid exploration of vastly larger
models than those currently in use.  As a first
step in this direction, we have explored a number of different
scaling axes, and shown that we can significantly outperform
a strong baseline generated with a 
state-of-the-art model.  Furthermore, the rapid
prototyping capabilities of our system enable experimentation
with many different model variants using a single representation
that is easy and efficient to generate.  We
explored some interesting characteristics of the hierarchical
phrase-based translation model using this capability, finding
that while longer phrases are no more helpful in the hierarchical
setting than in the standard setting, the use of discontiguous
phrases adds power to hierarchical models, in addition to acting
as a simplified type of lexicalized distortion.





