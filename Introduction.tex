\chapter{Introduction}\label{chap:introduction}

\begin{quote}
	{\em Computer Science is no more about computers than astronomy is about telescopes.}
	\begin{flushright}
		--Edsger Dijkstra
	\end{flushright}
\end{quote}
\begin{quote}
	{\em The purpose of computing is insight, not numbers.}
	\begin{flushright}
		--Richard Hamming
	\end{flushright}
\end{quote}

Machine translation is an old problem.  Like many other
computational problems it has been largely reinvented in 
the last fifteen years through the incorporation of machine 
learning methods.  The machine learning approach to machine
translation---called {\em statistical machine translation}---
requires considerable data and careful modeling.  
With available data increasing in size
and the best models increasing in complexity,
efficient algorithms are central to continued
progress.  This dissertation presents an algorithmic approach
to statistical machine translation that enables scaling to
large data and complex models, and fosters the rapid experimentation
needed to exploit them.

Most statistical machine translation systems implement a 
{\em direct representation} of the model.  This entails
offline computation of the complete model and storage of
this model in an efficient data structure in main memory 
at runtime.  In practice, these requirements
restrict the size of training data and the complexity
of the model, since both factors can affect model size.

\citet{Callison-Burch:2005:acl} and \citet{Zhang:2005:eamt}
proposed an alternative that we call {\em translation by pattern matching},
which we bring to fruition in this dissertation.  This
approach relies on an indirect representation of the model.
We store the training data itself in memory as its proxy.
Model parameters are then computed as needed at runtime.  This
representation is much more compact than the full model
and requires substantially less offline computation.  
It depends crucially on the ability to quickly search for 
relevant sections of the training data and compute
parameters as needed.   To perform this search, we need
algorithms for fast pattern matching on text, drawn primarily
from the literature in bioinformatics and information
retrieval.  Although
\citet{Callison-Burch:2005:acl} and \citet{Zhang:2005:eamt}
resolve several problems in the application of translation
by pattern matching to a standard phrase-based model, they leave behind several
open problems.  The first is whether their approach
can equal the performance of direct representation, 
a question that we answer affirmatively.

The next open problem we address is much harder.
\citet{Callison-Burch:2005:acl} and \citet{Zhang:2005:eamt}
applied their approach only to phrase-based translation models 
based on the translation of contiguous strings.  However,
a variety of translation models are now based on the 
translation of {\em discontiguous} phrases.  This new wrinkle
poses a difficult algorithmic challenge.
Exact pattern matching algorithms used for contiguous phrases don't work and
the best approximate pattern matching algorithms are much too slow,
taking several minutes per sentence.  A major contribution of
this dissertation is the development of new approximate 
pattern matching algorithms for translation models based on
discontiguous phrases.  Our algorithms reduce the empirical computation
time by two orders of magnitude, making translation by pattern matching
feasible for these models.

With these tools in hand, we address the problem of scaling
our translation model.  We then explore several scaling axes---some obvious and 
some novel---and show that we can significantly improve on 
a state-of-the-art hierarchical phrase-based translation system.  
Our experiments highlight interesting characteristics of the model
and interactions of various components.
Finally, we conclusively demonstrate that reproducing our results
with a direct representation would be extremely impractical.

\section{Outline of the Dissertation}

To set the stage for our research, Chapter~\ref{chap:survey} surveys the
current state of the art in statistical machine translation.  This is the
most extensive contemporary survey of the field.  It serves as a foundation 
on which the remaining chapters build.

Chapter~\ref{chap:overview} introduces translation by pattern matching.  We first
review previous work that serves as inspiration for our approach.  We then identify
some limitations of this work
and describe how we overcame these, transforming a clever idea into a 
practical solution.  We establish for the first time 
that translation by pattern matching is a
viable replacement for direct representation in a state-of-the-art phrase-based
translation model.

In Chapter~\ref{chap:algorithms} we apply translation by pattern matching to models
based on the translation of discontiguous strings.  This requires the invention 
of new pattern matching algorithms.  We describe our
algorithms in detail and analyze them theoretically and empirically.  Our algorithms
allow us to apply translation by pattern matching to hierarchical
phrase-based translation, a stronger baseline than the model used
in \ref{chap:overview}.  It matches the state of the art
achieved by direct representation.

In Chapter~\ref{chap:scaling} we apply translation by pattern matching to the problem of scaling
translation models.  We describe a number of axes along which
we can scale the models and investigate each of these empirically.  We then show
that our scaling methods improve translation quality over our strong
baseline.  Our experiments illuminate some interesting qualities
of these models and their interaction with aligned bilingual texts.  Finally,
we demonstrate that our results would be extremely difficult to replicate
using a direct representation.

Chapter~\ref{chap:conclusion} contains conclusions and future work.  We also
relate our work to other developments in the field, and describe
future research plans.

\section{Research Contributions}

The dissertation contains several important research contributions.

\begin{itemize}

	\item We describe a complete algorithmic framework for machine translation
	by pattern matching that enables scaling to large data, richer modeling,
	and rapid prototyping.

	\item We identify unanswered questions in the prior work that provides the foundation
	of our framework, and describe how we answer them to give
	state-of-the-art performance in standard phrase-based translation models.

	\item We introduce a novel approximate pattern matching algorithm that
	extends our framework to a variety of important models based on 
	discontiguous phrases.  Our algorithm solves an empirically difficult algorithmic problem.  
	It is two orders of magnitude faster than the previous state of the art, and 
	enables a number of previously difficult or impossible applications.

	\item We show how our framework can be applied to scale statistical translation
	models along several axes, leading to improvement in translation accuracy over
	a state-of-the-art baseline.

	\item We show that our method can scale to extremely large models, at least
	two orders of magnitude larger than most contemporary models.

	\item We include the most comprehensive survey of the field of statistical
	machine translation to date.
\end{itemize}







