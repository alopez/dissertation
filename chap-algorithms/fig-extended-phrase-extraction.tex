\begin{tikzpicture}
	\renewcommand\currentex[1]{\extractionex{5}{1, 4, 5}{2, 3}{1/1, 2/2, 3/3, 4/5, 5/4}{#1}}
	\currentex{
		\fill[main phrase] (source 2.north west) rectangle (target 3.south east);
		\fill[source gap] (source 4.north west) rectangle (source 4.south east);
	}
	\node [anchor=south,text width=5cm,text centered,above=11pt] at (source 3.north) {1. Adjacent word $f_4$ is candidate source gap.};
	\path (target 1.south west) -- +(0,-3) coordinate (ex 1 left);
	\begin{scope}[xshift=5cm]
		\currentex{
			\fill[main phrase] (source 2.north west) rectangle (target 3.south east);
			\fill[source gap] (source 4.north west) rectangle (source 4.south east);
			\fill[subtracted phrase] (target 4.north west) rectangle (target 5.south east);
		}
		\node [anchor=south,text width=5cm,text centered] at (source 3.north) {2. Find target span for gap.  Target span must be extended so that it is adjacent to main phrase.};
	\end{scope}
	\begin{scope}[xshift=10cm]
		\currentex{
			\fill[main phrase] (source 2.north west) rectangle (target 5.south east);
			\fill[subtracted phrase] (source 4.north west) rectangle (target 5.south east);
		}
		\node [anchor=south,text width=5cm,text centered] at (source 3.north) {3. Find reflected source span for gap.  Source span can increase as long as it is contiguous on one side of main phrase.};
		\draw[snake=brace,segment amplitude=2mm] (target 5.south east) -- (target 2.south west)
			node (extract label) [below=2mm,pos=0.5] {\fboxsep1pt 7. Extract $f_2 f_3 X_{\fbox{\scriptsize 1}} / e_2 e_3 X_{\fbox{\scriptsize 1}} $};
	\end{scope}
	\path (target 5.south east) -- +(0,-3) coordinate (ex 1 right);

	\path (extract label.south) -- +(0,-0.15) coordinate (bottom);
	\draw [gray,thin] (bottom) -- (bottom -| ex 1 left) coordinate (label loc);
	\draw [gray,thin] (bottom) -- (bottom -| ex 1 right);
	\node [anchor=south west] at (label loc) {Successful extension of phrase to include adjacent gap};


	% EXAMPLE TWO
	\begin{scope}[yshift=-4cm]
		\renewcommand\currentex[1]{\extractionex{5}{1, 5}{2,...,4}{1/4, 2/2, 3/2, 4/2, 5/3, 5/5}{#1}}
		\currentex{
			\fill[main phrase] (source 2.north west)
				-- (source 4.north east)
				-- (source 4.south east)
				-- (target 2.north east)
				-- (target 2.south east)
				-- (target 2.south west)
				--cycle;
			\fill[source gap] (source 5.north west) rectangle (source 5.south east);
		}
		\node [anchor=south,text width=5cm,text centered] at (source 3.north) {1. Adjacent word $f_5$ is candidate source gap.};
		\path (target 1.south west) -- +(0,-3) coordinate (ex 1 left);
		\begin{scope}[xshift=5cm]
			\currentex{
				\fill[main phrase] (source 2.north west)
					-- (source 4.north east)
					-- (source 4.south east)
					-- (target 2.north east)
					-- (target 2.south east)
					-- (target 2.south west)
					--cycle;
				\fill[source gap] (source 5.north west) rectangle (source 5.south east);
				\fill[subtracted phrase] (target 3.north west) rectangle (target 5.south east);
			}
			\node [anchor=south,text width=5cm,text centered,above=6pt] at (source 3.north) {2. Find target span for gap.};
		\end{scope}
		\begin{scope}[xshift=10cm]
			\currentex{
				\fill[main phrase] (source 2.north west)
					-- (source 4.north east)
					-- (source 4.south east)
					-- (target 2.north east)
					-- (target 2.south east)
					-- (target 2.south west)
					--cycle;
				\fill[subtracted phrase] (source 1.north west) 
					-- (source 5.north east)
					-- (target 5.south east)
					-- (target 3.south west)
					-- (target 3.north west)
					-- (source 1.south west)
					--cycle;
			}
			\node [anchor=south,text width=5cm,text centered] at (source 3.north) {3. Find reflected source span for gap.};
			\node (extract label) [anchor=north,text width=5cm,text centered] at (target 3.south) {\fboxsep1pt 7. Extraction fails because gap overlaps main phrase};
		\end{scope}
		\path (target 5.south east) -- +(0,-3) coordinate (ex 1 right);

		\path (extract label.south) -- +(0,-0.15) coordinate (bottom);
		\draw [gray,thin] (bottom) -- (bottom -| ex 1 left) coordinate (label loc);
		\draw [gray,thin] (bottom) -- (bottom -| ex 1 right);
		\node [anchor=south west] at (label loc) {Unsuccessful extension of phrase to include adjacent gap};
	\end{scope}


	% EXAMPLE THREE
	\begin{scope}[yshift=-8.5cm]
		\renewcommand\currentex[1]{\extractionex{5}{1, 5}{2,...,4}{1/4, 2/2, 3/2, 4/3, 5/5}{#1}}
		\currentex{
			\fill[main phrase] (source 2.north west)
				-- (source 4.north east)
				-- (source 4.south east)
				-- (target 3.north east)
				-- (target 3.south east)
				-- (target 2.south west)
				--cycle;
			\fill[source gap] (source 5.north west) rectangle (source 5.south east);
			\fill[source gap] (source 1.north west) rectangle (source 1.south east);
		}
		\node [anchor=south,text width=5cm,text centered,above=6pt] at (source 3.north) {1. Adjacent words $f_1$ and $f_5$ are candidate source gaps.};
		\path (target 1.south west) -- +(0,-3) coordinate (ex 1 left);
		\begin{scope}[xshift=5cm]
			\currentex{
				\fill[main phrase] (source 2.north west)
					-- (source 4.north east)
					-- (source 4.south east)
					-- (target 3.north east)
					-- (target 3.south east)
					-- (target 2.south west)
					--cycle;
				\fill[source gap] (source 5.north west) rectangle (source 5.south east);
				\fill[source gap] (source 1.north west) rectangle (source 1.south east);
				\fill[subtracted phrase] (target 4.north west) rectangle (target 4.south east);
				\fill[subtracted phrase] (target 5.north west) rectangle (target 5.south east);
			}
			\node [anchor=south,text width=5cm,text centered] at (source 3.north) {2. Find target span for each gap.  The combination of all spans must be contiguous.};
		\end{scope}
		\begin{scope}[xshift=10cm]
			\currentex{
				\fill[main phrase] (source 1.north west)
					-- (source 5.north east)
					-- (target 5.south east)
					-- (target 2.south west)
					-- (target 2.north west)
					-- (source 1.south west)
					--cycle;
				\fill[subtracted phrase] (source 1.north west)
					-- (source 1.north east)
					-- (source 1.south east)
					-- (target 4.north east)
					-- (target 4.south east)
					-- (target 4.south west)
					-- (target 4.north west)
					-- (source 1.south west)
					--cycle;
				\fill[subtracted phrase] (source 5.north west) rectangle (target 5.south east);
			}
			\node [anchor=south,text width=5cm,text centered,above=6pt] at (source 3.north) {3. Find reflected source span for gap.};
			\draw[snake=brace,segment amplitude=2mm] (target 5.south east) -- (target 2.south west)
				node (extract label) [below=2mm,pos=0.8] {\fboxsep1pt 7. Extract $X_{\fbox{\scriptsize 1}} f_2 f_3 f_4 X_{\fbox{\scriptsize 2}} / e_2 e_3 X_{\fbox{\scriptsize 2}} X_{\fbox{\scriptsize 1}}$};
		\end{scope}
		\path (target 5.south east) -- +(0,-3) coordinate (ex 1 right);

		\path (extract label.south) -- +(0,-0.15) coordinate (bottom);
		\draw [gray,thin] (bottom) -- (bottom -| ex 1 left) coordinate (label loc);
		\draw [gray,thin] (bottom) -- (bottom -| ex 1 right);
		\node [anchor=south west] at (label loc) {Extension to include both adjacent gaps};
	\end{scope}

\end{tikzpicture}
